\documentclass[a4paper]{article}

\usepackage[english]{babel}
\usepackage[utf8]{inputenc}
\usepackage[table]{xcolor} % colored tables
\usepackage{amsmath,graphicx,booktabs}
\usepackage[breaklinks=true,colorlinks]{hyperref}
\usepackage[colorinlistoftodos]{todonotes}
\usepackage{cite,caption,mathtools}

% partial derivative
\newcommand{\pp}[2]{\frac{\partial #1}{\partial #2}}
% in math mode: Description before symbol
\newcommand{\desc}[1]{\text{#1}\quad}
% abbrevations in math mode
\newcommand{\hydro}{\text{Hydro}}
\newcommand{\mhd}{\text{MHD}}
\newcommand{\adm}{\text{ADM}}

% left aligned vectors (pmatrix is center aligned)
\newenvironment{pvector}{\begin{pmatrix*}[l]}{\end{pmatrix*}}

% Defines pushright inside math mode
% source: https://tex.stackexchange.com/a/83522
\makeatletter
\newcommand{\pushright}[1]{\ifmeasuring@#1\else\omit\hfill$\displaystyle#1$\fi\ignorespaces}
\newcommand{\pushleft}[1]{\ifmeasuring@#1\else\omit$\displaystyle#1$\hfill\fi\ignorespaces}
\makeatother


\title{Advanced Hydrodynamics in ExaHyPE}
\author{Sven Köppel}
\date{2017-08-30}

\begin{document}
\maketitle

\begin{abstract}
In this document we review the formulations of general
relativistic hydrodynamics (GRHD) and general relativistic
magnetohydrodynamics (GRMHD) which shall be implemented in
ExaHyPE.
\end{abstract}

\tableofcontents

\section{Conservative PDEs}

We solve a conservation system
\begin{equation}\label{eq.cons.star}
\partial_\mu F^\mu=S
\quad\text{or}\quad
\partial_t F^0(u) + \partial_i F^i(u) = S(u)
\end{equation}
of the vector%
\footnote{We call these quantities \emph{vectors} even if they
are note vectors at all by the definition of a physicist, i.e.
they don't follow a transformation law. Instead, they are just
an (possibly) ordered set of variables}
of primitive variables $u$, the vector of converved variables
$F^0(u)=Q$, Fluxes $F^i(u)=F^i(Q)$ and Sources $S(u)=S(Q)$. In the
ExaHyPE termiology, this PDE reads
\begin{align}
\partial_t Q + \partial_i F^i(Q) &= S(Q), \\
\text{with indices:}\quad
\partial_t Q_j + \partial_i F^i_j(Q) &= S_j(Q).
\end{align}
Indeed in the ExaHyPE API, one provides the flux matrix
$F^i_j$ and the source vector $S_j$, with $i$ running over the
number of spatial dimensions (i.e. $i\in[0,1,2]$ in 3D) and $j$
running over the number of variables (size/dimension of the vector $Q$).

We can extend the conservation system with a nonconservative part.
The generalized conservation system reads
\begin{equation}
\pp {Q_j}t + \pp {F^i_j(Q)}{x^i} + B^{ij}(Q) \pp {Q_j}{x^i} = S_j(Q)
\end{equation}
where we introduced the new matrix $B^{ij}(Q)$ and refer to the
matrix multiplication $B^{ij}\partial_i Q_j$ as the
\emph{Nonconservative product}. In the following, we move the
nonconservative product to the algebraic source terms $S_j$
and refer to the sum as the \emph{right hand side} of the PDE,
i.e. we can define a new \emph{fused} source term
$S^*_j(Q, \partial_i Q_j)=S_j - B^{ij}\partial_i Q_j$
which absorbs the nonconservative product and resembles the
conservation system \eqref{eq.cons.star} we started with.

\section{GR Hydrodynamics}
The standard formulation of Relativistic hydrodynamics (Valencia)
uses
\begin{alignat}{4}
\label{eq.grhd.prim}
&\text{primitive variables}
&
\quad u &=
\begin{pvector}
\text{rest mass density} \\
\text{three-velocity} \\
\text{internal energy}
\end{pvector}
= \begin{pmatrix} \rho \\ v_i \\ \epsilon \end{pmatrix}
\\
\label{eq.grhd.cons}
&
\desc{conservative variables}
&
Q_\hydro
&=
\begin{pvector}
\text{Conserved density} \\
\text{Momentum} \\
\text{Rescaled energy density}
\end{pvector}
=
\begin{pmatrix} D \\ S_j \\ \tau \end{pmatrix}
\\
&
\desc{related by Prim2Cons}
&
Q(u)
&=
\begin{pvector}
\rho W \\
\rho h W^2 v_j \\
\rho h W^2 - p - \rho W
\end{pvector}
\label{eq.grhd.Qu}
\end{alignat}
\begin{alignat}{4}
&\text{with}
&
\desc{Lorentz factor}
W&=\alpha u^0 = (1 - v_i v^i)^{-1/2}
\\
&&
\desc{enthalpy} h &
\\
&&
\desc{pressure} p &= p(u) = p(\rho,\epsilon, v_i)
\\
&&
\desc{Total energy seen by Eulerian observer} E
&= \tau + D
\end{alignat}
In the literature, alternative namings are $W=\Gamma$, $E=U$.
The pressure is given by the equation of state, the \emph{closure}
equation of the system which governs the Prim2Cons operation.

Pay attention to whether you gain $v_i$ or $v^i$ from your Prim2Cons
operation. In \cite{THC} (and Lucianos book) they use $v_i$ while
\cite{BHAC} writes $v^i$. Some notations also state the primitive
variables as $u=(\rho,W v^i, p)$, i.e. pressure $p$ instead of
internal energy $\epsilon$.

\subsection{ADM variables}
The variables of the ADM formulation are the
\begin{equation}
\desc{ADM variables}
Q_\adm =
\begin{pmatrix}
\text{Lapse function} \\
\text{Shift vector} \\
\text{3-metric} \\
\text{extrinsic curvature}
\end{pmatrix}
=
\begin{pmatrix}
\alpha \\ \beta^i \\ \gamma_{ij} \\ K_{ij}
\end{pmatrix}.
\end{equation}
These variables play the role of \emph{material parameters}, i.e.
constants, when they enter the system of hydrodynamics.
Metric $\gamma_{ij}$ and extr. curvature $K_{ij}$ are symmetric,
only 6 degrees of freedom need to be stored. We can compute
$\gamma^{ij}$ from $\gamma_{ij}$ and use it for all raising and lowering
operations. $\gamma^{ij}$ is also only a 6-component object.
Remember that $\gamma^i_j=\delta^i_j$.

The hydrodynamic 3-energy-momentum-tensor is given by
\begin{equation}\label{eq.grhd.emtensor}
W^{ij} = S^i v^j + p \gamma^{ij}
\end{equation}
and is also refered to as $W^{ij}=S^{ij}$ in the literature and as
$W_\hydro^{ij}(p)$ in the subsequent chapters. Note that if you add terms
to the energy-momentum tensor (such as the magnetic terms), also $S_j$
will change (c.f. \eqref{eq.grhd.Qu} to XXX).

\subsection{Conformal treatment}
In the Valencia formulation, the whole system is written in terms
of tensor densities. The determinant of the metric
$\gamma=\text{det}(\gamma_{ij})$ relates the tensor densities with
an ordinary tensor. The 3-determinant is related to the determinant of
the four-metric by $\sqrt{-g}=\alpha\sqrt{\gamma}$. The system is
then formulated as
\begin{equation}
\partial_t (\sqrt{\gamma} Q)
+ \partial_i (\sqrt{\gamma} F^i) = \sqrt{\gamma} S.
\end{equation}
The treatment of the determinant resembles the conformal notations of
Einstein equations. In the following, we always denote fluxes and
sources \emph{without} the factor $\sqrt{\gamma}$. However, in case of
a conformal evolution, it has to be multiplied onto the fluxes and
sources.

\subsection{Fluxes}
The conservative formulation of the general relativistic hydrodynamics
is then given by the fluxes
\begin{align}
\label{eq.grhd.fluxes}
F^i_\hydro(u) &=
\begin{pmatrix}
\text{fluxes for } D \\
\text{fluxes for } S_j \\
\text{fluxes for } \tau
\end{pmatrix}
=
\begin{pvector}
D w^i \\
\alpha W^i_j - \beta^i S_j \\
\alpha (S^i - v^i D) - \beta^i \tau
\end{pvector}
\quad
\text{(general $W^{ij}$)}
\\
&=
\begin{pvector}
D w^i \\
S_j w^i + p \delta^i_j \\
\tau w^i + p v^i
\end{pvector}
\hspace{2.3cm} % todo: push right
\text{(\emph{only} for $W^{ij}=S^i v^i + p\gamma^{ij}$)}
\end{align}
%
Obviously, one can state the fluxes either by using the ADM variables
and the energy-momentum tensor and without. It is easy to proof that
the two ways of provide the flux are interchangable if \eqref{eq.grhd.emtensor}
holds.

It is convenient to write the fluxes with the vector
$w^i=\alpha v^i - \beta^i$ which is refered to as the advection
velocity relative to the coordinates, or just \emph{transport
velocity}. Some people avoid to introduce this vector since it is
not a real 3-vector. We introduce it only for abbreviation on the
paper and for saving contractions in the computer.

\subsection{Sources}
The Valencia sources \emph{with Christoffel symbols} are given by
\begin{equation}
S_\hydro(u) =
\begin{pvector}
\text{source for } D \\
\text{sources for } S_j \\
\text{source for } \tau
\end{pvector}
=
\begin{pvector}
0
\\
T^{\mu\nu} \partial_\mu g_{\nu j} - \Gamma^\delta_{\nu\rho} g_{\rho j} T^{\mu\nu}
\\
\alpha T^{\mu 0} \partial_\mu \ln \alpha - \alpha T^{\mu\nu} \Gamma^0_{\mu\nu}
\end{pvector}.
\end{equation}
In order to compute these sources, one needs to compute the
\begin{align}
\desc{Christoffel symbols of first kind} \Gamma^k_{ij}
&= \frac{1}{2} g^{kl} \left(\partial_i g_{jl} + \partial_j g_{il} - \partial_l g_{ij}\right)
\\
\desc{4-metric}
g_{\mu\nu} &= 
\begin{pmatrix}
-\alpha^2 + \beta_i \beta^i  & \beta_i \\
\beta_i                      & \gamma_{ij}
\end{pmatrix}
\\
\desc{4-energy-momentum-tensor}
T^{\mu\nu} &=
\rho h W^2 (n^\mu + v^\mu) (n^\nu + v^\nu)
\\ &\phantom{=} + p(\gamma^{\mu\nu} - n^\mu n^\nu)
\nonumber
\end{align}
One needs to recover the 4-velocity $u^\mu$ or the normal vector
$n^\mu$ in order to compute $T^{\mu\nu}$. All this is not desirable
at all, therefore one can derive the Christoffel-symbol free Source
terms for the Valencia formulation, as used in \cite{THC}. They read
\begin{equation}
S_\hydro(u) =
\begin{pvector}
0
\\
\frac \alpha2 S^{lm} \partial_j \gamma_{lm} + S_k \partial_j \beta^k - E \partial_j \alpha
\\
\alpha S^{ij} K_{ij} - S^i \partial_i \alpha
\end{pvector}
\end{equation}
These are the most general source terms for hydrodynamics on
\emph{dynamical} (curved) spacetime. In order to compute these sources,
we need the
\begin{align}
\desc{material parameters}  &= \{ \alpha, \beta^i, \gamma_{ij}, K_{ij} \}
\\
\desc{Nonconservative derivatives} &= \{ \partial_i \alpha,
\partial_i \beta^j, \partial_i \gamma_{jk} \}
\end{align}
The sources can be also easily written without computing the 3-energy-momentum-tensor,
exploiting
$\left(\partial_i \sqrt\gamma\right) / \sqrt\gamma = \frac 12 \gamma^{lm} \partial_i \gamma_{lm}$,
\begin{equation}
S_\hydro(u) =
\begin{pvector}
0
\\
\frac \alpha2 S^l v^m \partial_j \gamma_{lm} + \alpha p \frac 1{\sqrt{\gamma}} \partial_j \sqrt{\gamma} + S_k \partial_j \beta^k - E \partial_j \alpha
\\
\alpha S^l v^m K_{lm} - \alpha p \gamma^{lm} K_{lm} - S^i \partial_i \alpha
\end{pvector}
\end{equation}

\subsection{Cowling approximation}
In case of static spacetime (Cowling approximation,
$\partial_t \gamma_{ij} = 0$) one can simplify the source terms and
get rid of the contraction $S^{ij}K_{ij}$. Indeed, in this particular
case it is
\begin{equation}
S_{\hydro\text{,Cowling}}(u) =
\begin{pvector}
0
\\
\frac \alpha2 S^{lm} \partial_j \gamma_{lm} + S_k \partial_j \beta^k - E \partial_j \alpha
\\
\frac 12 S^{ik} \beta^j \partial_j \gamma_{ik} + S^j_i \partial_j \beta^i - S^i \partial_j \alpha
\end{pvector}
\end{equation}
This source term is \emph{not} valid for general spacetimes. In order to
compute these sources, one only needs
\begin{align}
\desc{material parameters}  &= \{ \alpha, \beta^i, \gamma_{ij} \}
\\
\desc{Nonconservative derivatives} &= \{ \partial_i \alpha,
\partial_i \beta^j, \partial_i \gamma_{jk} \}
\end{align}

\section{GR Magnetohydrodynamics}

The general relativistic magnetohydrodynamics GRMHD equations are
the consequence of the coupling of Euler equations 
(Hydrodynamics, GRHD) with Maxwell equations after applying the ideal MHD
approximation $\vec E = \vec B \times \vec v$. In this approximation,
the faraday tensor $F^{\mu\nu}$ (with 6 degrees of freedom, $\vec E$ 
and $\vec B$) reduces to the magnetic field $\vec B$ only, therefore the
vector of conserved variables in MHD is just $Q_\mhd = (B^i)$. We add
further evoluqion equations in case of the divergence cleaning technique.

\subsection{The GRMHD coupling}
In our conservative formulation, Euler (Hydrodynamics) and Maxwell (MHD)
couple with the pressure which get's replaced wherever it occurs in the
hydrodynamical equations derived above:
\begin{equation}
p \to p_\text{tot} = p_\hydro + p_\mhd
\quad\text{with}\quad
p_\mhd = \frac 12 \left( B_j B^j / W^2 + (B^j v_j)^2 \right).
\end{equation}
The total energy-momentum tensor is the sum of all involved theories,
thus the
\begin{alignat}{4}
& \desc{GRMHD EM-Tensor is}
& W^{ij} &= W^{ij}_\hydro(p_\hydro + p_\mhd) + W^{ij}_\mhd
\\
&\desc{with}
& W^{ij}_\mhd &= - B^i B^j / W^2 - (B^k v_k) v^i B^j
\\
&\desc{giving}
& W^{ij} &= 
S^i v^j + p_\text{tot} \gamma^{ij} - \frac{B^i B^j}{W^2} - (B^k v_k)v^i B^j
\end{alignat}
Consequently, also the conserved momentum changes to the 
\begin{alignat}{3}
\desc{GRMHD Prim2Cons}
&
Q(u)
&=
\begin{pvector}
D \\
S_j \\
\tau
\end{pvector}
&=
\begin{pvector}
\rho W \\
\rho h W^2 v_j + B^2 v_j - (B^i v_i) B_j \\
\rho h W^2 - p + \frac 12 \left( B^2 (1 + v^2) - (B^j v_j)^2 \right)
\end{pvector}
\label{eq.grhd.Qu}
\end{alignat}

\subsection{Fluxes and sources}
The evolution variables in \emph{pure} GRMHD, without mentioning any
magnetic field divergence treatment, are 
$Q = (Q_\hydro, Q_\mhd) = (D, S_j, \tau, B^j)$.
Note that we evolve the \emph{covariant} momentum $S_j$ and the
\emph{contravariant} magnetic field $B^j$. The vector of primitive
variables \eqref{eq.grhd.prim} does not change: There is no primitive
equivalent for the magnetic field.

The MHD conserved vector is just $Q_\mhd=(B^j)$ and has fluxes and sources
\begin{equation}
F^i(Q) = w^i B^j - B^i V^j,
\quad S(Q) = 0
\end{equation}
i.e. the new PDE for the magnetic field reads
\begin{equation}\label{eq.grmhd.pdeb.simple}
\partial_t B^j + \partial_i (w^i B^j - B^i V^j) = 0
\end{equation}
This evolution equation is \emph{only} used if one does \emph{not} employ
divergence cleaning. For instance, for constraint transport, the set
of PDEs is fully given.

%
%\subsection{The pure GRMHD equations}
%In consequence, the simple tensor-free fluxes of GRHD as given in
%\eqref{eq.grhd.fluxes} are no more correct. Therefore, one typically
%employs the fluxes with the energy-momentum-tensor. We repeat the first
%three entries for the GRHD and give provide the new flux for the magnetic
%field in the pure GRMHD (\emph{without} divergence cleaning) as the 
%fourth entry:
%\begin{equation}
%F^i(u) =
%\begin{pmatrix}
%\text{fluxes for }D \\
%\text{fluxes for }S_j \\
%\text{fluxes for }D \\
%\text{fluxes for }B^j
%\end{pmatrix}
%=
%\begin{pmatrix}
%w^i D
%\\
%\alpha W^i_j - \beta^i S_j
%\\
%\alpha (S^i - v^i D) - \beta \tau
%\\
%\end{pmatrix}.
%\end{equation}
%The source terms for the magnetic fields $B^j$ are zero, similar to the
%source terms for the density $D$.
%
%This formulation is useful if one uses a technique to control the 
%Maxwell constraint $\partial_j B^j = 0$ which does not change the
%PDE system, for instance contraint transport.

\subsection{The divergence cleaning (constraint damping) formulation}

With divergence cleaning, one introduces an additional field $\phi$
which is coupled to the magnetic field\footnote{Do not confuse the
symbol $\phi$ with the conformal factor of the BSSNOK/CCZ4 formulation}
and implements $\nabla B = - \kappa \phi$ as well as a radial advection
of $\phi$ with speed $\kappa$.
Hence, the PDE \eqref{eq.grmhd.pdeb.simple} is replaced by two
different PDEs.

The MHD conserved vector is $Q_\mhd = (B^j, \phi)$ and has
fluxes and sources
\begin{align}
%\desc{fluxes}
F^i(Q) &= \begin{pmatrix}
\text{fluxes for }B^j \\
\text{fluxes for }\phi
\end{pmatrix}
=
\begin{pvector}
w^i B^j - v^j B^i - B^i \beta^j \\
\alpha B^i - \phi \beta^i
\end{pvector}
\\
%\desc{and sources}
S(Q) &=
\begin{pmatrix}
\text{sources for } B^j \\
\text{sources for } \phi
\end{pmatrix}
=
\begin{pvector}
-B^i \partial_i \beta^j - \alpha \gamma^{ij} \partial_i \phi \\
- \alpha \kappa \phi - \phi \partial_i \beta^i
- \frac 12 \phi \gamma^{ij} \beta^k \partial_k \gamma_{ij}
+ B^i \partial_i \alpha
\end{pvector}
\end{align}

These terms come from \cite{BHAC} which derived it from
\cite{divclean1, divclean2}.

\subsection{Summary of the GRMHD+DivCleaning PDEs}
For the simple overview, we write down the full set of GRMHD equations
for \emph{dynamical spacetime} with divergence cleaning. The vector of
unknowns is $Q_\mhd=(D,S_j,\tau,B^i,\phi)$ and the fluxes and sources are
given by
\begin{align}
F^i_\mhd(Q) &=
\begin{pvector}
\text{fluxes for } D \\
\text{fluxes for } S_j \\
\text{fluxes for } \tau \\
\text{fluxes for }  B^j \\
\text{fluxes for } \phi
\end{pvector}
=
\begin{pvector}
w^i D \\
\alpha W^i_j - \beta^i S_j \\
\alpha (S^i - v^i D) - \beta^i \tau \\
w^i B^j - v^j B^i - B^i \beta^j \\
\alpha B^i - \phi \beta^i
\end{pvector}
\\
S_\mhd(Q) &=
S_\mhd
\begin{pvector}
D \\
S_j \\
\tau \\
 B^j \\
\phi
\end{pvector}
=
\begin{pvector}
0
\\
\frac \alpha2 S^{lm} \partial_j \gamma_{lm} + S_k \partial_j \beta^k - E \partial_j \alpha
\\
\alpha S^{ij} K_{ij} - S^i \partial_i \alpha
\\
-B^i \partial_i \beta^j - \alpha \gamma^{ij} \partial_i \phi \\
- \alpha \kappa \phi - \phi \partial_i \beta^i
- \frac 12 \phi \gamma^{ij}
\beta^k \partial_k \gamma_{ij}
+ B^i \partial_i \alpha
\end{pvector}
\end{align}
Obviously, we need the
\begin{alignat}{4}
&&
\desc{material parameters}  &= \{ \alpha, \beta^i, \gamma_{ij}, K_{ij} \}
\\
&\desc{and}
&
\desc{nonconservative derivatives} &= \{ \partial_i \alpha,
\partial_i \beta^j, \partial_i \gamma_{jk}, \partial_i \phi \}
\end{alignat}


\begin{thebibliography}{9}

\bibitem[THC]{THC}
  WhiskyTHC including the source terms
  \url{https://arxiv.org/abs/1312.5004}

\bibitem[BHAC]{BHAC}
  Porth, Olivares, Mizuno, Younsi, Rezzolla, Moscibrodzka, Falcke, Kramer 2017: The Black Hole Accretion Code
  \url{https://arxiv.org/pdf/1611.09720}

\bibitem[Palenzuela2008]{divclean1}
  Palenzuela, Lehner, Reula, Rezzolla 2008: 
  Beyond ideal MHD: towards a more realistic modeling of
  relativistic astrophysical plasmas
  \url{https://arxiv.org/pdf/0810.1838.pdf}
  
\bibitem[Dionysopoulou2013]{divclean2}
  Dionysopoulou, Alic, Palenzuela, Rezzolla, Giacomazzo 2013:
  General-relativistic resistive magnetohydrodynamics in
three dimensions: Formulation and tests
  \url{https://arxiv.org/pdf/1208.3487.pdf}

\end{thebibliography}


\end{document}