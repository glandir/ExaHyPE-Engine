\documentclass[a4paper]{article}

\usepackage[english]{babel}
\usepackage[utf8]{inputenc}
\usepackage[table]{xcolor} % colored tables
\usepackage{amsmath,graphicx,booktabs}
\usepackage[breaklinks=true,colorlinks]{hyperref}
\usepackage[colorinlistoftodos]{todonotes}
\usepackage{cite,caption,mathtools}

% partial derivative
\newcommand{\pp}[2]{\frac{\partial #1}{\partial #2}}
% in math mode: Description before symbol
\newcommand{\desc}[1]{\text{#1}\quad}
% abbrevations in math mode
\newcommand{\hydro}{\text{Hydro}}
\newcommand{\mhd}{\text{MHD}}
\newcommand{\adm}{\text{ADM}}

% left aligned vectors (pmatrix is center aligned)
\newenvironment{pvector}{\begin{pmatrix*}[l]}{\end{pmatrix*}}

% Defines pushright inside math mode
% source: https://tex.stackexchange.com/a/83522
\makeatletter
\newcommand{\pushright}[1]{\ifmeasuring@#1\else\omit\hfill$\displaystyle#1$\fi\ignorespaces}
\newcommand{\pushleft}[1]{\ifmeasuring@#1\else\omit$\displaystyle#1$\hfill\fi\ignorespaces}
\makeatother


\title{Nonconservative EulerFlow}
\author{Sven Köppel}
\date{2017-10-24}

\begin{document}
\maketitle

\begin{abstract}
In this document, the mathematics of the nonconservative
EulerFlow is proposed.
\end{abstract}

\tableofcontents

\section{Conservative PDEs}

We solve a conservation system
\begin{equation}\label{eq.cons.star}
\sum_\mu
\partial_\mu F^\mu=S
\quad\text{or}\quad
\partial_t F^0(u) + \partial_i F^i(u) = S(u)
\end{equation}
of the vector%
\footnote{We call these quantities \emph{vectors} even if they
are note vectors at all by the definition of a physicist, i.e.
they don't follow a transformation law. Instead, they are just
an (possibly) ordered set of variables}
of primitive variables $u$, the vector of converved variables
$F^0(u)=Q$, Fluxes $F^i(u)=F^i(Q)$ and Sources $S(u)=S(Q)$. In the
ExaHyPE termiology, this PDE reads
\begin{align}
\partial_t Q + \partial_i F^i(Q) &= S(Q), \\
\text{with indices:}\quad
\partial_t Q_j + \sum_i \partial_i F^i_j(Q) &= S_j(Q).
\end{align}
Indeed in the ExaHyPE API, one provides the flux matrix
$F^i_j$ and the source vector $S_j$, with $i$ running over the
number of spatial dimensions (i.e. $i\in[0,1,2]$ in 3D) and $j$
running over the number of variables (size/dimension of the vector $Q$).

We can extend the conservation system with a nonconservative part.
The generalized conservation system reads
\begin{equation}
\pp {Q_j}t + \sum_i \pp {F^i_j(Q)}{x^i} + \sum_{i,j} B^{ik}_j(Q) \pp {Q_k}{x^i} = S_j(Q)
\end{equation}
where we introduced the new matrix $B^{ij}_k(Q)$ and refer to the
matrix multiplication $B^{ij}\partial_i Q_j$ as the
\emph{Nonconservative product}. In the following, we move the
nonconservative product to the algebraic source terms $S_j$
and refer to the sum as the \emph{right hand side} of the PDE,
i.e. we can define a new \emph{fused} source term
$S^*_j(Q, \partial_i Q_j)=S_j - B^{ik}_j\partial_i Q_k$
which absorbs the nonconservative product and resembles the
conservation system \eqref{eq.cons.star} we started with.

\section{Classical Euler Equations}

The classical hydrodynamic equations are defined with the state vector
\begin{equation}
Q = (\rho, S_j, E)
\end{equation}
where $S_j$ is the $j$th entry of the momentum vector $\vec S = \rho \vec v$ with $\vec v$ the velocity of the fluid, $\rho$ the mass density of the fluid and $E$ the energy of the fluid.

Euler equations need an equation of state, we impose here
\begin{equation}
p = (\gamma-1) \left( E - \frac { \vec S^2 }{2 \rho} \right)
\end{equation}
with typical values $\gamma=1.4$ and $\vec S^2 = \sum_i S^i S^i$.

\subsection{Conservative formulation}

The conservative system is given by the PDE $\partial_t Q + \nabla F(Q) = 0$ with

\begin{equation}\label{fluxes}
F_i = F_i\begin{pvector} \rho \\ S_j \\ E \end{pvector}
=
\begin{pvector}
S_i \\
S_i S_j / \rho + p \delta_{ij} \\
S_i (E + p) / \rho
\end{pvector}
\end{equation}

where $\delta_{ij}$ is the Kronecker delta ($\delta_{ij}=1$ if $i=j$, else 0).

\subsection{Nonconservative formulation}

The Euler equations can also be casted as a fully nonconservative system with zero Fluxes but with a nonzero $B\nabla Q$. It is then
\begin{equation}
\partial_t Q_k + \sum_i \partial_i F^i(Q) =
\partial_t Q_k + \sum_{i,j} B^{ik}_j \partial_i Q_k
\end{equation}
That is, the nonconservative product is just given by the sum of 
the fluxes over all directions:
\begin{equation}
B \nabla Q = \sum_{i=0}^3 \pp{F^i(Q)}{x^i}
\end{equation}
To come up with a $B\nabla Q$, we thus have to analytically derive
the fluxes \eqref{fluxes}:
\begin{equation}\label{BgradQ}
\partial_i F^i =
\begin{pvector}
\partial_i S_i
\\
\left( \partial_i 1/\rho \right) S_i S_j
+ (\partial_i S_i) S_j / \rho
+ S_i (\partial_i S_j) / \rho
+ (\partial_i p) \delta_{ij}
\\
\left( \partial_i 1/\rho \right) S_i (E+p)
+ 1/\rho (\partial_i S_i) (E+p)
+ 1/\rho S_i (\partial_i E + \partial_i p)
\end{pvector}
\end{equation}
Since the Engine gives us only the derivatives of the state variables, i.e. $\nabla Q = \{ \partial_i Q_j \}$, we have to resolve the definitions of
\begin{align}\label{helpers}
\partial_i \rho^{-1} &= - (\partial_i \rho) / \rho^2
\\
\partial_i p &= (\gamma-1) \left( \partial_i E
 - \left( \partial_i \rho^{-1} \vec S^2 \right)
 - \frac{\sum_a S_a \partial_i S^a}{\rho} \right)
\end{align}
Summing up, with equations \eqref{BgradQ} and \eqref{helpers}, we
can implement the nonconservative product for the $i$th state
variable $Q_i$ simply as
\begin{equation}
B(Q) \nabla Q = \sum_{d=1}^3 \partial_d F^d
\end{equation}
If one has a convenient notation for accessing the gradient matrix and the state vector, it's quite straightforward to implement this system.


\end{document}
